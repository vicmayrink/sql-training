\usepackage{xcolor}
\usepackage{xcolor-material}
\usepackage[utf8]{inputenc}
\usepackage[default]{lato}
\usepackage[T1]{fontenc}
\usepackage{lmodern}
\usepackage{listingsutf8}
\usepackage{hyperref}
\usepackage{wrapfig}
\usepackage{dsfont}
\usepackage[export]{adjustbox}
\usepackage{array, booktabs}
\usepackage{cellspace}
\usepackage{realboxes}
\usepackage{xpatch}
\usepackage[most]{tcolorbox}

% colors
\definecolor{color1}{HTML}{607D8B}
\definecolor{color2}{HTML}{333333}

\graphicspath{{./resources}}
\lstset{inputpath=./resources, xleftmargin=7mm, xrightmargin=7mm}

\definecolor{codegreen}{rgb}{0,0.6,0}
\definecolor{codegray}{rgb}{0.5,0.5,0.5}
\definecolor{mygray}{rgb}{0.5,0.5,0.5}
\definecolor{codepurple}{rgb}{0.58,0,0.82}
\definecolor{backcolour}{rgb}{0.95,0.95,0.92}
\hypersetup{colorlinks,urlcolor=blue}

\newcommand{\redhighlight}[1]{{\color{red}\textbf{\texttt{#1}}}}
\newcommand{\bluehighlight}[1]{{\color{orange}\textbf{\texttt{#1}}}}

\newcommand{\R}{\mathds{R}}
\newcommand{\appallingunderline}[1]{
	\underline{\smash{#1}\vphantom{T}}\vphantom{#1}%
}

\lstdefinestyle{sql}{
	backgroundcolor=\color{backcolour},
	commentstyle=\color{codegreen},
	keywordstyle=\color{magenta},
	numberstyle=\tiny\color{codegray},
	stringstyle=\color{codepurple},
	basicstyle=\ttfamily\scriptsize,
	breakatwhitespace=false,
	breaklines=true,
	captionpos=b,
	keepspaces=true,
	numbers=left,
	numbersep=5pt,
	showspaces=false,
	showstringspaces=false,
	showtabs=false,
	tabsize=2
}

\lstdefinestyle{winprompt}{
	backgroundcolor=\color{backcolour},
	commentstyle=\color{codegreen},
	keywordstyle=\color{magenta},
	numberstyle=\tiny\color{codegray},
	stringstyle=\color{codepurple},
	basicstyle=\ttfamily\footnotesize,
	breakatwhitespace=false,
	breaklines=false,
	breakatwhitespace=false,
	prebreak=false,
	captionpos=b,
	keepspaces=true,
	showspaces=false,
	showstringspaces=false,
	showtabs=false,
	tabsize=1
}

\lstdefinestyle{codeinline}{
	backgroundcolor=\color{gray},
	commentstyle=\color{codegreen},
	keywordstyle=\color{magenta},
	numberstyle=\tiny\color{codegray},
	stringstyle=\color{codepurple},
	basicstyle=\ttfamily,
	breakatwhitespace=false,
	breaklines=true,
	captionpos=b,
	keepspaces=true,
	numbers=left,
	numbersep=5pt,
	showspaces=false,
	showstringspaces=false,
	showtabs=false,
	tabsize=2
}

\lstset{style=sql}
\lstset{inputencoding=utf8/latin1}

%Inline code
\newcommand{\code}[1]{
	\tcbox[on line, boxsep=1.5pt, left=0pt,right=0pt,top=0pt,
		bottom=0pt,colback=backcolour,boxrule=0pt]{
			{\lstinline[style=codeinline]|#1|}
		}
}

%%% fancy boxes
\usepackage{tcolorbox}
\usepackage{wrapfig}
\def\fullboxbegin{
\bigskip
\begin{tcolorbox}[colback=color1,colframe=color1,coltext=white,arc=0mm,boxrule=0pt]
}
\def\fullboxend{\end{tcolorbox}\medskip}
%
\def\leftboxbegin{
\begin{wrapfigure}{l}{0.5\textwidth}
\begin{tcolorbox}[colback=color1,colframe=color1,coltext=white,arc=0mm,boxrule=0pt]
}
\def\leftboxend{
\end{tcolorbox}
\end{wrapfigure}
}
%
\def\rightboxbegin{
\begin{wrapfigure}{r}{0.5\textwidth}
\begin{tcolorbox}[colback=color1,colframe=color1,coltext=white,arc=0mm,boxrule=0pt]
}
\def\rightboxend{
\end{tcolorbox}
\end{wrapfigure}
}
%

\newcounter{reasonbox}
\def\reasonboxbegin#1#2{
	\bigskip
	\refstepcounter{reasonbox}
	\begin{tcolorbox}[colback=white,colframe=#2,arc=0mm,title={\textbf{#1}}]
}
\def\reasonboxend{
	\end{tcolorbox}
}

\newcounter{highlightboxbegin}
\def\highlightboxbegin#1#2#3{
	\medskip
	\refstepcounter{highlightboxbegin}
	\begin{centering}
	\begin{tcolorbox}[colback=white,colframe=#2,arc=0mm,title={\textbf{#1}},width=#3\linewidth]
}
\def\highlightboxend{
	\end{tcolorbox}
	\end{centering}
}

%%%

%Footnote with no markers
%Reference: https://tex.stackexchange.com/a/30726
\newcommand\blfootnote[1]{%
  \begingroup
  \renewcommand\thefootnote{}\footnote{\vspace{-0.2\baselineskip}\fontsize{4}{6}\selectfont#1}%
  \addtocounter{footnote}{-1}%
  \endgroup
}

%Self hyperref
\newcommand\hrefself[1]{\href{#1}{#1}}

\makeatletter
\xpatchcmd{\@footnotetext}{\@parboxrestore}{\@parboxrestore\leftskip-1in}{}{}
\xpatchcmd{\footnoterule}{\hrule}{\llap{\smash{\rule[-.4pt]{7mm}{.4pt}}}\hrule}{}{}
\makeatother

%Source: https://tex.stackexchange.com/a/16362
\tikzset{
  every overlay node/.style={
    %draw=black,fill=white,rounded corners,
	anchor=north west,
  },
}
% Usage:
% \tikzoverlay at (-1cm,-5cm) {content}; or
% \tikzoverlay[text width=5cm] at (-1cm,-5cm) {content};
\def\tikzoverlay{%
   \tikz[baseline,overlay]\node[every overlay node]
}



\addtobeamertemplate{footnote}{\hspace{-8mm}}{}