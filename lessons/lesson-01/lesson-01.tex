\documentclass[t, 10pt, aspectratio=169, table, x11names]{beamer}
\usetheme{metropolis}

\usepackage{xcolor}
\usepackage{xcolor-material}
\usepackage[utf8]{inputenc}
\usepackage[default]{lato}
\usepackage[T1]{fontenc}
\usepackage{lmodern}
\usepackage{listingsutf8}
\usepackage{hyperref}
\usepackage{wrapfig}
\usepackage{dsfont}
\usepackage[export]{adjustbox}
\usepackage{array, booktabs}
\usepackage{cellspace}
\usepackage{realboxes}
\usepackage{xpatch}
\usepackage[most]{tcolorbox}

% colors
\definecolor{color1}{HTML}{607D8B}
\definecolor{color2}{HTML}{333333}

\graphicspath{{./resources}}
\lstset{inputpath=./resources, xleftmargin=7mm, xrightmargin=7mm}

\definecolor{codegreen}{rgb}{0,0.6,0}
\definecolor{codegray}{rgb}{0.5,0.5,0.5}
\definecolor{mygray}{rgb}{0.5,0.5,0.5}
\definecolor{codepurple}{rgb}{0.58,0,0.82}
\definecolor{backcolour}{rgb}{0.95,0.95,0.92}
\hypersetup{colorlinks,urlcolor=blue}

\newcommand{\redhighlight}[1]{{\color{red}\textbf{\texttt{#1}}}}
\newcommand{\bluehighlight}[1]{{\color{orange}\textbf{\texttt{#1}}}}

\newcommand{\R}{\mathds{R}}
\newcommand{\appallingunderline}[1]{
	\underline{\smash{#1}\vphantom{T}}\vphantom{#1}%
}

\lstdefinestyle{sql}{
	backgroundcolor=\color{backcolour},
	commentstyle=\color{codegreen},
	keywordstyle=\color{magenta},
	numberstyle=\tiny\color{codegray},
	stringstyle=\color{codepurple},
	basicstyle=\ttfamily\scriptsize,
	breakatwhitespace=false,
	breaklines=true,
	captionpos=b,
	keepspaces=true,
	numbers=left,
	numbersep=5pt,
	showspaces=false,
	showstringspaces=false,
	showtabs=false,
	tabsize=2
}

\lstdefinestyle{winprompt}{
	backgroundcolor=\color{backcolour},
	commentstyle=\color{codegreen},
	keywordstyle=\color{magenta},
	numberstyle=\tiny\color{codegray},
	stringstyle=\color{codepurple},
	basicstyle=\ttfamily\footnotesize,
	breakatwhitespace=false,
	breaklines=false,
	breakatwhitespace=false,
	prebreak=false,
	captionpos=b,
	keepspaces=true,
	showspaces=false,
	showstringspaces=false,
	showtabs=false,
	tabsize=1
}

\lstdefinestyle{codeinline}{
	backgroundcolor=\color{gray},
	commentstyle=\color{codegreen},
	keywordstyle=\color{magenta},
	numberstyle=\tiny\color{codegray},
	stringstyle=\color{codepurple},
	basicstyle=\ttfamily,
	breakatwhitespace=false,
	breaklines=true,
	captionpos=b,
	keepspaces=true,
	numbers=left,
	numbersep=5pt,
	showspaces=false,
	showstringspaces=false,
	showtabs=false,
	tabsize=2
}

\lstset{style=sql}
\lstset{inputencoding=utf8/latin1}

%Inline code
\newcommand{\code}[1]{
	\tcbox[on line, boxsep=1.5pt, left=0pt,right=0pt,top=0pt,
		bottom=0pt,colback=backcolour,boxrule=0pt]{
			{\lstinline[style=codeinline]|#1|}
		}
}

%%% fancy boxes
\usepackage{tcolorbox}
\usepackage{wrapfig}
\def\fullboxbegin{
\bigskip
\begin{tcolorbox}[colback=color1,colframe=color1,coltext=white,arc=0mm,boxrule=0pt]
}
\def\fullboxend{\end{tcolorbox}\medskip}
%
\def\leftboxbegin{
\begin{wrapfigure}{l}{0.5\textwidth}
\begin{tcolorbox}[colback=color1,colframe=color1,coltext=white,arc=0mm,boxrule=0pt]
}
\def\leftboxend{
\end{tcolorbox}
\end{wrapfigure}
}
%
\def\rightboxbegin{
\begin{wrapfigure}{r}{0.5\textwidth}
\begin{tcolorbox}[colback=color1,colframe=color1,coltext=white,arc=0mm,boxrule=0pt]
}
\def\rightboxend{
\end{tcolorbox}
\end{wrapfigure}
}
%

\newcounter{reasonbox}
\def\reasonboxbegin#1#2{
	\bigskip
	\refstepcounter{reasonbox}
	\begin{tcolorbox}[colback=white,colframe=#2,arc=0mm,title={\textbf{#1}}]
}
\def\reasonboxend{
	\end{tcolorbox}
}

\newcounter{highlightboxbegin}
\def\highlightboxbegin#1#2#3{
	\medskip
	\refstepcounter{highlightboxbegin}
	\begin{centering}
	\begin{tcolorbox}[colback=white,colframe=#2,arc=0mm,title={\textbf{#1}},width=#3\linewidth]
}
\def\highlightboxend{
	\end{tcolorbox}
	\end{centering}
}

%%%

%Footnote with no markers
%Reference: https://tex.stackexchange.com/a/30726
\newcommand\blfootnote[1]{%
  \begingroup
  \renewcommand\thefootnote{}\footnote{\vspace{-0.2\baselineskip}\fontsize{4}{6}\selectfont#1}%
  \addtocounter{footnote}{-1}%
  \endgroup
}

%Self hyperref
\newcommand\hrefself[1]{\href{#1}{#1}}

\makeatletter
\xpatchcmd{\@footnotetext}{\@parboxrestore}{\@parboxrestore\leftskip-1in}{}{}
\xpatchcmd{\footnoterule}{\hrule}{\llap{\smash{\rule[-.4pt]{7mm}{.4pt}}}\hrule}{}{}
\makeatother

%Source: https://tex.stackexchange.com/a/16362
\tikzset{
  every overlay node/.style={
    %draw=black,fill=white,rounded corners,
	anchor=north west,
  },
}
% Usage:
% \tikzoverlay at (-1cm,-5cm) {content}; or
% \tikzoverlay[text width=5cm] at (-1cm,-5cm) {content};
\def\tikzoverlay{%
   \tikz[baseline,overlay]\node[every overlay node]
}



\addtobeamertemplate{footnote}{\hspace{-8mm}}{}

\begin{document}

	\author{Victor Mayrink}
	\title{Curso de SQL}
	\subtitle{Aula 1: Instalação das ferramentas e configuração do ambiente}

	\begin{frame}[plain]
		\maketitle
	\end{frame}
	
	\begin{frame}
		\frametitle{Resumo da aula}
		\vspace{0.5cm}
		\begin{figure}[h]
			\includegraphics[width=0.70\textwidth]{docker-window.png}
		\end{figure}
	\end{frame}
	
	\begin{frame}
		\frametitle{Requisitos}
		Para fazer atividades do curso, vamos precisar instalar algumas ferramentas
		\begin{columns}[t]
			\begin{column}{0.20\textwidth}
				\vspace{0.3cm}
				\begin{figure}[h]
					\includegraphics[width=1.5cm, right]{docker.png}
				\end{figure}
				\vspace{1.1cm}
				\begin{figure}[h]
					\includegraphics[width=1.5cm, right]{dbeaver.png}
				\end{figure}
			\end{column}

			\begin{column}{0.70\textwidth}
				\begin{itemize}
					\item \textbf{Docker}: \href{https://www.docker.com/products/docker-desktop/}{\appallingunderline{https://www.docker.com/products/docker-desktop/}}
					
					\begin{itemize}
						\item Utilizaremos o Docker para simular um \textit{servidor} de banco de dados no nosso próprio computador
						\item Além do Docker, vamos precisar da \textit{imagem} do banco de dados que iremos executar
					\end{itemize}
					\vspace{0.4cm}
					\item \textbf{Dbeaver}: \href{https://dbeaver.io/download}{ \appallingunderline{https://dbeaver.io/download}}
					\begin{itemize}
						\item Interface amigável para executar comandos no banco de dados
						\item Suporta diferentes tipos de bancos de dados
					\end{itemize}
				\end{itemize}
			\end{column}
			\begin{column}{0.05\textwidth}
			\end{column}
		\end{columns}
	\end{frame}
	
	\begin{frame}
		\frametitle{Instalação do Docker}
		\begin{enumerate}
			\item Vá até \href{https://www.docker.com/products/docker-desktop/}{\appallingunderline{https://www.docker.com/products/docker-desktop/}}
			\item Baixe o instalador do \textit{Docker Desktop} compatível com seu sistema
			\item Siga os passos de instalação
		\end{enumerate}
		\begin{figure}[h]
			\fbox{\includegraphics[width=0.40\textwidth]{docker-webpage.png}}
			\hspace{0.5cm}
			\fbox{\includegraphics[width=0.355\textwidth]{docker-installation.png}}
		\end{figure}
	\end{frame}
	
	\begin{frame}
		\frametitle{Docker}
		\vspace{0.5cm}
		\begin{figure}[h]
			\includegraphics[width=0.70\textwidth]{docker-window.png}
		\end{figure}
	\end{frame}
	
	\begin{frame}
		\frametitle{Instalação Dbeaver}
		\vspace{0.5cm}
		\begin{figure}[h]
			\includegraphics[width=0.45\textwidth]{dbeaver-install-start.png}
			\hspace{0.5cm}
			\includegraphics[width=0.45\textwidth]{dbeaver-install-finish.png}
		\end{figure}
	\end{frame}
	
	\begin{frame}
		\frametitle{Instalação Dbeaver}
		\vspace{0.5cm}
		\begin{figure}[h]
			\includegraphics[width=0.70\textwidth]{dbeaver-window.png}
		\end{figure}
	\end{frame}
	
	\begin{frame}
		\frametitle{Microsoft PowerShell}
		Usuários do Windows também devem instalar o \textbf{Microsoft PowerShell}
		\vspace{0.6cm}
		\begin{columns}
			\column{0.07\textwidth}
			\column{0.15\textwidth}
			\begin{figure}[h]
				\includegraphics[width=1.5cm]{powershell-icon.png}
			\end{figure}
			\column{0.80\textwidth}
			\begin{enumerate}
				\item Abra o prompt de comando e execute o winget para instalar o Microsoft PowerShell
				\lstinputlisting[style=winprompt]{winget-powershell.cmd}
				\vspace{0.4cm}
				\item Utilize a barra de pesquisas e procure pelo PowerShell. Verifique se o programa foi instalado corretamente.
			\end{enumerate}
			\column{0.60\textwidth}
		\end{columns}
	\end{frame}
	
	\begin{frame}
		\frametitle{Microsoft PowerShell}
		\vspace{0.5cm}
		\begin{figure}[h]
			\includegraphics[width=0.70\textwidth]{powershell-window.png}
		\end{figure}
	\end{frame}
	
	\begin{frame}
		\frametitle{Instalação das ferramentas}
		\vspace{0.5cm}
		\begin{center}
			\LARGE
			Muito bem, finalizamos a instalação das ferramentas! :D
		\end{center}
		\vspace{1cm}
		\begin{center}
			\LARGE
			Agora vamos iniciar a configuração! :(
		\end{center}
	\end{frame}
	
	\begin{frame}[t]
		\frametitle{Executando o MySQL com o database SakilaDB}
		Vamos utilizar o Docker para emular um banco de dados na nossa própria máquina.
		
		Clique na barra de pesquisa do Docker Desktop e procure pela imagem \textbf{\texttt{sakiladb/mysql}}
		\vspace{0.3cm}
		\begin{figure}[h]
			\includegraphics[width=0.45\textwidth]{docker-sakila-search-1.png}
			\hspace{0.5cm}
			\includegraphics[width=0.45\textwidth]{docker-sakila-search-2.png}
		\end{figure}
	\end{frame}
	
	\begin{frame}[t]
		\frametitle{Executando o MySQL com o database SakilaDB}
		Ao executar o container do banco \textbf{\texttt{sakiladb}}, os logs são exibidos na tela.
		
		Os logs devem indicar quando o servidor \texttt{\textbf{MySQL}} estiver pronto para receber novas conexões.
		\vspace{0.1cm}
		\begin{figure}[h]
			\includegraphics[width=0.55\textwidth]{docker-sakiladb-ready.png}
		\end{figure}
	\end{frame}

	\begin{frame}[t]
		\frametitle{Conectando-se ao banco através do DBeaver}
		Agora vamos nos conectar ao banco de dados através do Dbeaver.
		
		\begin{enumerate}
			\small
			\item Abra o Dbeaver e clique em \texttt{Arquivo > Novo}
			\item Em seguida, selecione \texttt{Conexão com banco de dados}, dento do diretório Dbeaver
			\item Escolha o banco \texttt{\textbf{MySQL}} e clique em avançar
		\end{enumerate}
		
		\vspace{0.1cm}
		\begin{figure}[h]
			\includegraphics[width=0.32\textwidth]{dbeaver-new.png}
			\hspace{0.5cm}
			\includegraphics[width=0.32\textwidth]{dbeaver-db-select.png}
		\end{figure}
	\end{frame}
	
	\begin{frame}
		\frametitle{Conectando-se ao banco através do DBeaver}
		\vspace{0.05cm}
		\begin{columns}[t]
			\begin{column}{0.50\linewidth}
				\centering
				\begin{enumerate}
					\small
					\setcounter{enumi}{3}
					\item Preencha as credenciais de conexão, conforme a documentação da imagem do \textbf{\texttt{sakiladb}} que está em execução.
					\begin{itemize}
						\item \textbf{host:} localhost
						\item \textbf{port:} 3306
						\item \textbf{database:} sakila
						\item \textbf{user:} sakila
						\item \textbf{password:} p\_ssw0rd
					\end{itemize}
					\vspace{0.15cm}
					\item Finalmente clique em \texttt{Testar Conexão}
				\end{enumerate}
			\end{column}
			\begin{column}{0.45\linewidth}
				\begin{figure}[h]
					\includegraphics[width=0.85\textwidth, left]{dbeaver-edit-connection.png}
				\end{figure}
			\end{column}
		\end{columns}
	\end{frame}
	
\end{document}